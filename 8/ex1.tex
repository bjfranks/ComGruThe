% !TeX root = root.tex
% !TeX spellcheck = en_US
\section{Decision problem vs generating set computation}

\subsection{The Extended Set Transporter problem polynomially reduces to the Set Transporter problem.}

Obviously the Set Transporter problem can be solved using the Extended Set Transporter Problem we simply ignore the $\alpha$ and $\beta$.

Now for the other direction we inductively reduce the Extended Set Transporter Problem to itself, but reduce the number of $\alpha$ and $\beta$ in each step.

Given a group $G$, $\Delta$, $\Delta'$, $\alpha_1,\dots,\alpha_t$ and $\beta_1, \dots, \beta_t$. We consider $\alpha_t$ and $\beta_t$, we check whether there is an element $g \in G$, with $\alpha^g=\beta$, using Schreier-Sims. We now calculate $G_{\alpha}$ using Schreier-Sims and use the Extended Set Transporter Problem with $G_{\alpha}$, $\Delta$, $\Delta'^{g^{-1}}$, $\alpha_1,\dots,\alpha_{t-1}$ and $\beta_1^{g^{-1}}, \dots, \beta_{t-1}^{g^{-1}}$.

If we assume we find an $h \in G_{\alpha}$ with the requirements, then $hg$ is an element with the before requirement $\Delta^{hg}=(\Delta^h)^g=(\Delta'^{g^{-1}})^g=\Delta'$, the same for all the $\alpha$. The other direction can be argued similarly.

\subsection{The Set Stabilizer problem polynomially reduces to the Extended Set Transporter problem}

Given $G$ and $\Delta$, we wish to find $G_{(\Delta)}$. Using the Extended Set Transporter problem we can find the orbits of elements in $G_{(\Delta)}$, by checking whether $\exists g \in G$ with $(\Delta/\{\alpha\})^g=(\Delta/\{\beta\})$ and $\alpha^g=\beta$. If we repeat this for all $\alpha$ and $\beta$ we get the orbits in $\mathcal{O}(n^2)$. Now we can also get the corresponding transversals by reapplying the Extended Set Transporter problem, this can be done in conjunction with finding orbits in $\mathcal{O}(n^4)$. Using these methods we can effectively apply the Schreier-Sims algorithm. Which then gives us the required $G_{(\Delta)}$

\subsection{The Set Transporter problem polynomially reduces to the Set Stabilizer problem}

Given $G$, $\Delta$ and $\Delta'$. We consider the domain $\Omega \times \{0,1\}$ and let $G$ act independently on $\Omega \times \{0\}$ and $\Omega \times \{1\}$, we add an element to this group that interchanges $\Omega \times \{0\}$ and $\Omega \times \{1\}$, via an identity, i.e. $\phi(a,x)=(a,1-x)$. We call this group $\hat{G}$

Now if we count $|G_{(\Delta)}|$ and $|G_{(\Delta')}|$, we can count $|\hat{G}_{(\Delta\times \{0\} \cup \Delta'\times \{1\})}|$. If there exists no $g \in G$ for which $\Delta^g=\Delta'$, then the size will be equal to $|G_{(\Delta)}||G_{(\Delta')}|$, however if such a $g$ exists then the size will be larger, since the $\phi$ from above can be used in conjuntion with this $g$ to produce more elements fixing $(\Delta\times \{0\} \cup \Delta'\times \{1\})$.

