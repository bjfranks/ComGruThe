% !TeX root = root.tex
% !TeX spellcheck = en_US
\section{Group sudoku}

\subsection*{a) Fill out the tables so they yield non-isomorphic groups. Argue that your groups are non-isomorphic.}

\begin{tabular}{|c|c|c|c|c|c|c|c|}\hline
e & z & a & b & c & d & f & g \\\hline
z & e & c & f & a & g & b & d \\\hline
a & d & f & c & z & b & g & c \\\hline
b & f & d & e & g & a & z & c \\\hline
c & g & b & a & e & f & d & z \\\hline
d & a & z & g & f & e & c & b \\\hline
f & b & g & z & d & c & e & a \\\hline
g & c & e & d & b & z & a & f \\\hline
\end{tabular}

This is effectively $D_4$, with $e=id,z=(13),a=(1234),b=(24),c=(14)(23),d=(12)(34),f=(13)(24),g=(1432)$

\begin{tabular}{|c|c|c|c|c|c|c|c|}\hline
e & z & a & b & c & d & f & g \\\hline
z & e & f & g & d & c & a & b \\\hline
a & f & z & c & g & b & e & d \\\hline
b & g & d & z & a & f & c & e \\\hline
c & d & b & f & z & e & g & a \\\hline
d & c & g & a & e & z & b & f \\\hline
f & a & e & d & b & g & z & c \\\hline
g & b & c & e & f & a & d & z \\\hline
\end{tabular}

This is effectively $Dic_2$, with $e=e,z=a^2,a=a,b=b,c=ab,d=a^3b,f=a^3,g=a^2b$

These groups are clearly not isomorphic, because in $D_4$ almost every element is its inverse, wheras in $Dic_2$ only $z$ is its own inverse. This yields the two groups have different subgroup orders and cannot be isomorphic.

\newpage
\subsection*{b) Show that there are groups $G$ and $H$ such that $Z(G) \cong Z(H)$ and $G/Z(G) \cong H/Z(H)$ but $G$ and $H$ are non-isomorphic}

$Z(D_4)=\{e,f\}$, $Z(Dic_2)=\{e,z\}$. They are clearly isomorphic.

$D_4/Z(D_4)$, where $e=\{e,f\}, a = \{a,g\}, b=\{b,z\}, c= \{c,d\}$ looks like this:

\begin{tabular}{|c|c|c|c|}\hline
e & a & b & c \\\hline
a & e & c & b \\\hline
b & c & e & a \\\hline
c & b & a & e \\\hline
\end{tabular}

$Dic_2/Z(Dic_2)$, where $e=\{e,z\}, a = \{a,f\}, b=\{b,g\}, c= \{c,d\}$ looks like this:

\begin{tabular}{|c|c|c|c|}\hline
e & a & b & c \\\hline
a & e & c & b \\\hline
b & c & e & a \\\hline
c & b & a & e \\\hline
\end{tabular}

clearly they are isomorphic, this together with a) proves the claim.