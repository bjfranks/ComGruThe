% !TeX root = root.tex
% !TeX spellcheck = en_US
\section{(Semi-)regular actions}

\subsection*{a)}
Let G semi-regular. Assume $H < G$ and $H$ transitive. Let $g \in G/H$ and let $\alpha in \Omega$. From G being semi-regular we can follow $\alpha \neq \alpha^g = \beta \in Omega$. Since H is transitive $\exists h \in H:\alpha^h=\beta$, however $\alpha^gh^{-1}=\alpha$ and $gh^{-1} \neq id$ which contradicts $G$ being semi-regular. Thus $H$ is either not transitive or $H=G$.

\subsection*{b)}
Claim: All finite primitive regular permutation groups are generated by elements whose cycles intersect in a pairwise fashion only in one element, such that the generated group is transitive.

Every element of the group may not fix any $\alpha$. There can not be two elements mapping $\alpha^g=\beta$. Any element of $Sym(\Omega)$ is a union of cycles. Say we have two elements $g \neq h \neq g^t \forall t \in \mathbb{N}$. Then there must be an element $\alpha$, s.t. $\alpha^{<g>} \neq \alpha^{<h>}$, i.e. two intersection cycles in g and h. Also $\alpha^{<g>} \cap \alpha^{<h>} = \{\alpha\}$, otherwise we could fix $\alpha$ via the additional element. From this know any two cycles in $g$ and $h$ intersect in only one element.

Examples: $<(12)(34),(14)(23)>, <(123)(456),(14)(25)(36)>$

Nonexample: $<(12)(34)(56)(78),(14)(23)(58)(67)>$

\subsection*{c)}