% !TeX root = root.tex
% !TeX spellcheck = en_US
\section{Free groups and presentations}
\subsection*{a)}
Let $X, Y$ be two sets with the same cardinality and $F(X), F(Y)$ the corresponding free groups. Let $\varphi : X \rightarrow Y$ be a bijection. We define the maps $\varphi_1 : X \rightarrow F(Y), x \mapsto \varphi(x)$ and $\varphi_2 : Y \rightarrow F(X), y \mapsto \varphi^{-1}(y)$. By the universal property there exist unique homomorphisms $\varphi_1^\prime : F(X) \rightarrow F(Y)$ s.t. $\varphi_1^\prime(x) = \varphi_1(x)$ for all $x \in X$ and $\varphi_2^\prime : F(Y) \rightarrow F(X)$ s.t. $\varphi_2^\prime(y) = \varphi_2(y)$ for all $y \in Y$.

We show that $\varphi_2^\prime \circ \varphi_1^\prime = id$. Let $x \in F(X)$ be arbitrary. As $F(X)$ is generated by $X$, we have $x = x_1 \cdots x_r$ for $x_i \in X$. Then, it holds that
\[ \varphi_2^\prime \circ \varphi_1^\prime(x) = \varphi_2^\prime(\varphi_1^\prime(x_1) \cdots \varphi_1^\prime(x_r)) = \varphi_2^\prime(\varphi_1^\prime(x_1)) \cdots \varphi_2^\prime(\varphi_1^\prime(x_r)) = x_1 \cdots x_r = x. \]
Thus, $\varphi_1^\prime$ is an isomorphism from $F(X)$ to $F(Y)$.


\subsection*{b)}
Let $F(X)$ be a free group and $x, y \in F(X)$. We show that $x$ and $y$ commute if and only if there exists $z \in F(X)$ s.t. $x = z^n$ and $y = z^m$ for some $n, m \in \mathds{N}$.

The second direction is clear. For the first direction assume $x$ and $y$ commute. Thus, the subgroup $\langle x,y \rangle \leq F(X)$ is abelian. By the lecture we know that subgroup of free groups are free. Hence, $\langle x,y \rangle$ is a free abelian group. But then $\langle x,y \rangle \cong \mathds{Z}$ since $\mathds{Z}$ is the only free abelian group. This means that $\langle x,y \rangle$ is cyclic and therefore generated by some $z \in F(X)$. It follows that $x = z^n$ and $y = z^m$ for some $n, m \in \mathds{N}$.

\subsection*{c)}
Consider the group $G = \langle a,b \mid a^{-1}bab^{-2}, b^{-1}aba^{-2} \rangle$. In $G$ we have $ab = baa$ and $ba = abb$. By plugging in we get $ab = abba$ and $ba = baab$ which implies that $ab = ba = e$. Therefore, we have
\[ e = ab = baa = ea = a \]
and
\[ e = ba = abb = eb = b. \]
This implies that $G = \{e\}$.
