% !TeX root = root.tex
% !TeX spellcheck = en_US
\section{Sylow Theorems}

\subsection*{a) Characterize all simple, abelian groups}
Let $G$ be a simple, abelian group.

First we show that $G$ is cyclic and generated by every non-trivial element.\\
Let $e \neq g \in G$. Then $\langle e \rangle \subsetneq \langle g \rangle \trianglelefteq G$ since every subgroup of $G$ is normal as $G$ is abelian. Thus, $\langle g \rangle = G$ since $G$ is simple.

If $|G| = \infty$, then $G \cong \mathds{Z}$ by 1.3.2.

Assume $|G| < \infty$.

If $|G| = n$ for some $n \in \mathds{N}$, we show that $n$ is prime.\\
By Cauchy (1.8.1) for every prime divisor $p$ of $n$ there is an element of order $p$. Since $G$ is generated by any non-trivial element, it follows that $p = n$.

Therefore, $G \cong \mathds{Z}/p$ for some prime number $p \in \mathds{N}$.

Lastly, we show that $\mathds{Z}/p$ is simple, abelian for any prime number $p \in \mathds{N}$.\\
Clearly, $\mathds{Z}/p$ is abelian. It is also simple as $|\mathds{Z}/p| = p$ is prime and by Lagrange the order of every subgroup divides $p$. Thus, $\langle 0 \rangle$ and $\mathds{Z}/p$ are the only (normal) subgroups of $\mathds{Z}/p$.

Thus, the finite simple, abelian groups are exactly $\mathds{Z}/p$ for some prime number $p \in \mathds{N}$.

\subsection*{b) There is no simple group of order 1070}
Let $G$ be a group with $|G| = 1070 = 2 \cdot 5 \cdot 107$. By the 3\textsuperscript{rd} Sylow Theorem for the number $n_5$ of Sylow-5-groups of $G$ it holds that $n_5 \mid 2 \cdot 107$ and $n_5 \equiv 1 \text{ mod } 5$. Thus, $n_5 = 1$. This means that there is exactly one Sylow-5-group $H \leq G$ of order 5. By the 2\textsuperscript{nd} Sylow Theorem $H$ is in a singleton conjugacy class and therefore $H$ is a non-trivial normal subgroup of $G$. It follows that $G$ is not simple. 
